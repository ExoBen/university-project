
\documentclass[../dissertation.tex]{subfiles}
\begin{document}

\chapter{Introduction}

Rendering large networks of data (tens or hundreds of thousands of nodes and edges) in a browser is a frequent problem encountered in visualisation software due to limits in browser performance and the ability to render the information in a meaningful way the user can make sense of it. This project aims to analyse many different software packages available and outline different approaches which can be taken to help minimize the clutter and performance required to display these large networks. 
 
\section{Context}
Currently, the problem is that far too much data is passed to the client from the server, both requireing a lot of bandwidth and processing power, and once the data is finally rendered, the result is a mass of nodes of which no useless information can be taken from. 

The ideal end result of the project would involve taking the data from the server that would normally be graphed, and instead analysing that data, and then passing a modified version of the data to the client. This could be in the form of an image, removing unnecessary nodes / edges, or node bundling / edge bundling. This would result in the information being far more useful to the client, with them being able to make useful decisions based on the network presented to them, as opposed to before where they were shown a huge mass of nodes that could not easily deciphered. The end result would also ideally reduce load times / CPU / RAM required. 

useful? 

Visualisation of large networks is challenging for a number of reasons, incuding the amount of processing power and memory required in order to produce the network or get an interactive visualisation of it, and the difficulty of displaying the network clearly so a user can get useful information from it. Limiting the research to visualising the data in a browser adds another level of complexity as browsers perform far more poorly than servers or average computers and there are far fewer choices of software available that will run in a browser.

\section{Aims and Objectives}



\section{Achievements}



% \subsection{A subsection}
% The quick brown fox jumped over the lazy dog.

\end{document}