
\documentclass[../dissertation.tex]{subfiles}
\begin{document}

\chapter{Introduction}

Rendering massive networks (tens or hundreds of thousands of nodes and edges) in a browser is a frequent problem encountered in visualisation software due to limits in browser performance and the lack of ability to render the information in a meaningful way that the user can make sense of it. This dissertation aims to analyse many different software packages available and outline different approaches which can be taken to help minimize the clutter and performance required to display these large networks, and then to describe, analyse and evaluate a system created for the purpose of this project.

There are several barriers to easily visualising massive amounts of information, including the huge amount of data that needs to be passed to the client from the server, requiring a lot of bandwidth, processing power and RAM, and, once the data is finally rendered, the result is a mass of nodes of which no useless information can be taken from. 

A successful outcome of the project would be a system where, instead of the data being passed straight to the client that is to be visualised, that data is analysed, and a modified and reduced version of the data is sent to the client. This could be done by: sending an image to the client; removing unnecessary nodes/edges; or node bundling/edge bundling. This would result in the information being far more useful to the client, with them being able to make informed decisions based on the network presented to them, as opposed to before where they were shown a huge mass of nodes that could not be easily deciphered. This would also reduce the load times of the network, resulting in networks displaying far faster, and less bandwidth and processing requirements on the client's side.

\end{document}