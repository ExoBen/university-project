\documentclass[../dissertation.tex]{subfiles}
\begin{document}

\chapter{Motivation}

Towards the end of my summer internship with SAS, I started discussing some possible project ideas that interested both them and me. Visualisation of huge networks in a browser was one of the ideas that we discussed which was interesting for a number of reasons. SAS already has in-house network visualisation software which works fine for up to a few thousand nodes, but begins to struggle in several ways for many more nodes than that. On top of that, network visualisation is something that I was interested in and seemed like something that I could work on that would stimulate me and be of benefit to SAS as well.

There were several directions the project could be taken in, from improving the performance of the current software so that it could handle far more data and still load (relatively) quickly, to trying to do more processing server side so that the client had less to do, to joining nodes or edges together that are similar (bundling) in order to reduce loading times for the client. Additionally, an important choice to make was whether or not time is something that mattered. A possible way to view the project would be to come with a way to display the networks as efficiently as possible. However, for huge networks, there is no point in rendering all of the nodes if they don't show anything useful due to the sheer number of them. Hence, another possible take on the project would be, with or without time constraints, to come up with a way to allow users to visualise huge networks (based on) hundreds of thousands or even millions of nodes and gain valuable knowledge.

TODO: expand - make a paragraph on time vs visualisabilty and also why displaying all is not plausible. 

During the first couple of weeks of the project, contact was made to both my adviser and SAS in regards to what direction the project should be taken in. The end result of this discussion was that, at least initially, research would be focused on finding effective ways of visualising large networks, with time not being a critical factor, i.e. focusing on the visualisabilty of the network as opposed to trying to make it render as fast as possible. 

\end{document}