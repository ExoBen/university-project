\documentclass[../dissertation.tex]{subfiles}
\begin{document}

\chapter{Motivation}

There is limited research and development of software in regards to the visualisation of massive networks in a browser. A solution for companies who require network visualisation can be to use existing software (whether open source or with a license attached), or to develop their own software. SAS \cite{sas} have developed their own network visualisation software (for example, within the SAS product, Visual Investigator \cite{sasvi}) which performs well for up to a few thousand nodes, but begins to struggle when the number of nodes exceeds that - in relation to processing power, system memory and bandwidth. As a result, SAS, who explicitly support this project, are keen to find ways for larger networks to be visualised both with high performance and high information throughput.

The solution to this problem is either to enhance the existing software or develop new solutions. If the current software was to be improved, then the goal would be to find out what bottlenecks exist throughout the process of loading a network, and then try to minimise the impact of the bottlenecks. If new solutions were to be created, then there are several possible ways the software's performance could be improved:

\begin{itemize}
	\item The network could be processed server-side, before it is passed to the client in order to:
	\begin{itemize}
	    \item Put less stress on the user's machine, so the transfer of the network and its rendering becomes quicker.
	    \item Make visualisation clearer, by removing unnecessary information.
	\end{itemize}
	\item Joining nodes or edges together that are similar (bundling) could be done in order to reduce loading times. This could be done on either the client or server-side.
	\item Different tools could be explored that could make visualisations clearer to the user
\end{itemize}

Following discussion with peers and then conducting initial research, it was found that there was little research done into massive network visualisation. Csardi and Nepusz have stated that there is a "lack of network analysis software which can handle large graphs efficiently, and can be embedded into a higher level program or programming language (like Python, Perl or GNU R)" \cite{csardi2006igraph}. Another article, by Chen and Chaomei, declared that there is "A prolonged lack of low-cost, ready-to-use, and reconfigurable information visualization systems" \cite{chen2005top}. 

There are two important metrics to consider for massive network visualisation: the time it will take to visualise the networks from the user asking for it, and the amount of useful information the user will get from the displayed network. If neither of these criteria are met then the visualisation software is unusable. If the system is slow but displays useful information at the end then the system is not very user friendly, but can still be useful. Finally, if the system is both fast and displays useful information then it an optimal network visualisation system.

\end{document}