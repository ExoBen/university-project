\documentclass[../dissertation.tex]{subfiles}
\begin{abstract}

Visualising networks with thousands of nodes and edges in a browser causes both developers and users numerous difficulties. Challenges include the high amount of bandwidth and computational power required to render a massive network, along with the difficulty of getting useful information from the network, due to its complexity, after it is rendered. After completing background research into the field, a systematic review of existing network visualisation software was conducted. This found that Tulip was the best overall network visualisation system that was reviewed, based on its performance, documentation and techniques for handling the visualisation of massive networks. Then, a network visualisation system was designed and implemented with the goal of manipulating networks in order to reduce their size and then visualise these networks. This was done by removing as much unnecessary content as possible while keeping the structure of the network intact. Finally, the system was tested and its performance was evaluated. It was found that applying node pruning followed by node bundling based on number of edges resulted in a 99.997\% decrease in loading times. The performance evaluation confirmed that node bundling techniques improve performance of network visualisation systems significantly and additionally result in the network being visualised more clearly.

\end{abstract}